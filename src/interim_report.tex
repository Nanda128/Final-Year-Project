\documentclass[conference]{IEEEtran}
\IEEEoverridecommandlockouts
% The preceding line is only needed to identify funding in the first footnote. If that is unneeded, please comment it out.

\usepackage{cite}
\usepackage{amsmath,amssymb,amsfonts}
\usepackage{algorithm}
\usepackage{graphicx}
\usepackage{textcomp}
\usepackage{xcolor}
\newcommand{\BibTeX}{\textrm{B \kern -.05em \textsc{i \kern -.025em b} \kern -.08em
T \kern -.1667em \lower .7ex \hbox{E} \kern -.125emX}}
\begin{document}

    \title{Digital Twin Framework for Autonomous Drone Swarm Coordination in Maritime SAR Operations}

    \author{\IEEEauthorblockN{Nandakishore Vinayakrishnan}
    \IEEEauthorblockA{\textit{Department of Computer Science and Information Systems} \\
    \textit{University of Limerick}\\
    Limerick, Ireland \\
    23070854@studentmail.ul .ie / 0009-0009-7390-5955}}

    \maketitle

    \begin{abstract}
        Maritime Search and Rescue operations rely on effective coordination and perfect readiness among several assets.
        This project proposes a modular Digital Twin framework to enable autonomous drone swarms to coordinate GPS-denied SAR missions in maritime environments.
        The framework seeks to integrate real-time sensor data, digital-physical system synchronization, and Human-in-the-Loop (HITL) interactions to enhance situational awareness and decision-making.
        A simulation-based approach will be employed to validate that framework, focusing on swarm coordination, Human-Computer Interaction principles, and overall situational awareness.
        This work seeks to address the critical gap between centralized SAR systems and decentralized autonomous operations, with the aim of improving response times and increasing readiness by enabling the utilization of more economic assets.
    \end{abstract}

    \begin{IEEEkeywords}
        Digital Twin, Autonomous Unmanned Aerial Vehicles (UAVs), Swarm Coordination, Maritime Search and Rescue, GNSS-Denied Environments, Human-in-the-Loop, Real-time Data Integration, Human-Computer Interaction, Cyber-Physical Systems, Real-time Simulation
    \end{IEEEkeywords}

    \section{Introduction}\label{sec:introduction}
        In 2025,

    \bibliographystyle{plain}
    \bibliography{interim_report}

\end{document}
